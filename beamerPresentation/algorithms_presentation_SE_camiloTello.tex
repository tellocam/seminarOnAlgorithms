\documentclass{beamer}
%\documentclass[aspectratio=169]{beamer}

% You have to install the theme first!

% corporate design of TU Wien
\usetheme[font=liberation]{tuw}
% background "TU main building" on title page
%\usetheme[tuw_background]{tuw}
% individual background on title page
%\usetheme[tuw_image=TU_Background]{tuw}
% white logo if you have a dark background image
%\usetheme[tuw_image=TU_Background,tuw_whitelogo]{tuw}
% sidebar (not in TU Wien CD! but nice for long presentations)
% width of the sidebar can be changed with option: "width=2cm"
%\usetheme[outer=sidebar]{tuw}
% move frametitle up (beside logo)
%\usetheme[tuw_frametitletotop]{tuw}

% if you use german umlaute use T1 encoding:
%\usepackage[T1]{fontenc}
% default Latex fonts are not T1 supported -> bitmaps used, this is not nice on
% screen; you can use the lmodern package instead
%\usepackage{lmodern}
\usepackage[utf8]{inputenc}
\usepackage{listings}
\usepackage{booktabs}
\usepackage{url}

%%% title page settings
\title[TUW Theme Demo]{%
  TU Wien Theme Demo
}
\subtitle{This demo presentation shows and tests the TU Wien LaTeX beamer
  theme.}
\author{OSUE-Team 2016}
\date{\today}

% optional institute (used in other outer themes; not part of TU Wien's CD)
\institute[TU Wien]
{%
  Institut für Technische Informatik\\
  Technische Universität Wien\\
  ---\\
  182.709 Operating Systems UE\\
  WS 2016/17
}

%%% slides start here
\begin{document}

% first frame must include the title page!
\begin{frame}
  \titlepage
\end{frame}

% table of contents if you have a long presentation (uses 'part' and 'section'
% elements)
\begin{frame}{Outline}
  \tableofcontents
\end{frame}

\section{Elements}

\subsection[Paragraphs]{Simple Paragraphs}
\begin{frame}[fragile]
  \frametitle{Title first category}
  \framesubtitle{Title second category}
  % no deeper title hierarchy provided

  Donec suscipit luctus lacus ut viverra. Proin molestie eros tellus, vitae
  elementum nulla fringilla nec. Pellentesque facilisis, elit ac egestas
  gravida, ante leo euismod velit, et suscipit est ex ut ex.

  You can cite~\cite{Tan11}. Urls look like this: \url{http://www.google.com/}.
\end{frame}

\begin{frame}[fragile]{Emphasize}
  \begin{lstlisting}
This sentence includes \emph{emphasized}
and \alert{alerted} words.
  \end{lstlisting}
  becomes:

  This sentence includes \emph{emphasized} and \alert{alerted} words.
\end{frame}

\subsection{Blocks}
\begin{frame}{Blocks}
  \begin{block}{Block}
    This is a block.
  \end{block}
  \begin{exampleblock}{Example}
    This is an example block.
  \end{exampleblock}
  \begin{alertblock}{Attention}
    This is an alert block.
  \end{alertblock}
\end{frame}

\subsection{More Blocks}
\begin{frame}{More Blocks}{theorem, proof}
  \begin{theorem}
    This is a theorem.
  \end{theorem}
  \begin{proof}
    This is a proof.  Donec suscipit luctus lacus ut viverra. Proin molestie
    eros tellus, vitae elementum nulla fringilla nec. Pellentesque facilisis,
    elit ac egestas gravida, ante leo euismod velit, et suscipit est ex ut ex.
  \end{proof}
\end{frame}

\subsection{Lists}
\begin{frame}{Lists}{itemize, enumerate, description}
  \begin{itemize}
  \item Sed semper egestas libero vitae tempus.
    \begin{itemize}
    \item Second level.
      \begin{itemize}
      \item Third level.
      \end{itemize}
    \end{itemize}
  \end{itemize}

  \begin{enumerate}
  \item Sed semper egestas libero vitae tempus.
  \item Interdum et malesuada fames ac ante ipsum primis in faucibus.
  \end{enumerate}

  \begin{description}
  \item[Sed] semper egestas libero vitae tempus.
  \item[Interdum] et malesuada fames ac ante ipsum primis in faucibus.
  \end{description}
\end{frame}

\begin{frame}{Figures}{}
  \begin{figure}
    \centering
    \includegraphics[width=0.8\linewidth]{TU_Logo}
    \caption{Exemplary figure (\texttt{TU\_Logo.pdf}).}
    \label{fig:test}
  \end{figure}
\end{frame}

\begin{frame}{Tables}{}
  \begin{table}
    \centering
    \begin{tabular}{lcr}
      \toprule
      Column 1 & Column 2 & Column 3 \\
      \midrule
      Some Data & a & 1234 \\
      Some More & b & 1331 \\
      Here We Go & ab & 12 \\
      \bottomrule
    \end{tabular}
    \caption{Exemplary table.}
    \label{tbl:example}
  \end{table}
\end{frame}

\section{Special}

\subsection{Columns}
\begin{frame}{Columns}
  \begin{columns}[c]
    \column{.5\textwidth}
    \begin{itemize}
    \item Some text
    \item with a picture on the right
    \item blah blah
    \item more blah blah
    \end{itemize}
    \column{.5\textwidth}
    \includegraphics[width=.8\textwidth]{TU_Logo}
  \end{columns}
\end{frame}

\subsection{Code}
\begin{frame}[fragile]{Code}
   \begin{lstlisting}[basicstyle=\ttfamily]
#include <stdio.h>

int main(void)
{
  (void) printf("Hello, C World\n");

  return 0;
}
   \end{lstlisting}

   Note: Don't forget to mark the frames using listings with
   \lstinline[columns=fixed]{[fragile]}.
\end{frame}

\subsection{Footnotes}
\begin{frame}{Footnotes}
  You can also place footnotes, e.g.,
  here~\footnote{This is a footnote.}
  and here~\footnote{This is a longer footnote going over two lines.
    So I've added some more blah blah. Lorem ipsum whatever.}.
\end{frame}

\section{References}
\begin{frame}{References}
  You can add references like in a LaTeX article.

  \begin{thebibliography}{10}
  \bibitem{Tan11}[1] T. Tantau, J. Wright and V. Miletić: The Beamer class -
    User Guide for version 3.36, 2011, available online:
    \url{http://texdoc.net/texmf-dist/doc/latex/beamer/doc/beameruserguide.pdf}.
  \end{thebibliography}
\end{frame}


\end{document}
