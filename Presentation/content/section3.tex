\section{Heuristics}

\subsection{NP - Completeness}

\begin{frame}[fragile]{3DS-IVC $\in$ NP ?}
  \begin{overlayarea}{\textwidth}{\textheight}
    \only<1->{
      \begin{Lemma}<1->
        3DS-IVC $\in$ NP
      \end{Lemma}
    }
    
    \only<2->{
      \begin{block}{Proof.}
        \only<2->{
          Given a solution for 3DS-IVC, encoded with pairs of integers which are bounded between 0 and
          $\sum_{v \in V} w(v_i)$ $\mapsto$ are solutions in polynomial space. \\
        }
        \only<3->{
          Verify that no adjacent edges have overlapping intervals. $\forall (u, v) \in E:$
          \[ \left[ \text{{start}}(u), \text{{start}}(u) + w(u) \right) \cap \left[ \text{{start}}(v), \text{{start}}(v) + w(v) \right) = \emptyset. \]
        }
        \only<4->{
          Since $|E| \leq \frac{n(n-1)}{2}$ and checking if two intervals overlap is in $\mathcal{O}(1)$,
          the verification can be done in $\mathcal{O}(n^2)$ \qed.
        }
      \end{block}
    }
  \end{overlayarea}
\end{frame}

\begin{frame}{NAE-3SAT $\propto$ 3DS-IVC ?}
  \begin{block}{Idea of Proof}
    \begin{itemize}
      \onslide<1->{\item Construct an instance 3DS-IVC from an instance of NAE-3SAT in polynomial time. \\}
      \onslide<2->{\item Verify that a positive instance of NAE-3SAT results in a positive instance of 3DS-IVC.\\}
      \onslide<3->{\item Verify that if the created instance of 3DS-IVC is positive, then the instance of NAE-3SAT is also positive.\\}
      \onslide<4->{\item Since the 3DS-IVC problem is in NP and is harder than NAE-3SAT, which is an NP-Complete problem, 3DS-IVC is NP-Complete \qed.}
    \end{itemize}
  \end{block}
  
  \onslide<5->{
    The proof yields:
      \onslide<5->{
        \begin{theorem}
          Deciding whether a 27-pt stencil can be colored with less than $k$ colors is NP-Complete. \cite{main_paper}
        \end{theorem}
      }
  }
\end{frame}

\subsection{Algorithms}
\begin{frame}{Algorithms: Bipartite Decomposition}
  \begin{block}<1->{Recall: Theorem}
    For bipartite graphs there exists a coloring \\ 
  $\text{maxcolor*} = \max_{(i,j) \in E} w(i) + w(j)$ attainable in $\mathcal{O}(|E|)$.
  \end{block}

  \begin{definition}<2->
    Let $c(x, y)$ be the lower end of the
    color interval associated with vertex $(x, y)$ in that coloring.
    And let $\text{RC} = \max c(x, y) + w(x, y)$ be the maximum color used by any of the rows.
  \end{definition}

  \begin{corollary}<3->
    RC $\leq$ maxcolor* is a lower bound of the optimal number of colors of the instance since it is the
    optimal coloring of a subgraph of the original instance.
  \end{corollary}

\end{frame}

\begin{frame}{Algorithms: Bipartite Decomposition}
  \only<1->{
  \begin{algorithm}[H]
    \TitleOfAlgo{Bipartite Decomposition}
    \scriptsize
    \SetAlgoNlRelativeSize{0}
    
    % \caption*{Algorithm: Bipartite Decomposition}
    
    \KwData{2DS-IVC instance with Y rows}
    \KwResult{Approximate coloring with at most \(2 \cdot \text{maxcolor}^*\)}
    
    \For{each row $r$ in the 2DS-IVC instance}{
        \If{$r$ is even}{
            Color the chain of vertices in row $r$ optimally using colors from $[0, RC)$\;
        }
        \Else{
            Color the chain of vertices in row $r$ optimally using colors from $[RC, 2RC)$\;
        }
    }
    
    \Return{Approximate coloring with at most \(2 \cdot \text{maxcolor}^*\)}
    
    \end{algorithm}
    
  }

    \begin{theorem}<2->
      Since the coloring obtained by the algorithm is at most $2 \cdot \text{maxcolor}^*$, the Bipartite Decomposition is a 2-approximation
      algorithm for 2DS-IVC. \cite{main_paper}
    \end{theorem}

    \begin{theorem}<3->
      the Bipartite Decomposition is a 4-approximation algorithm for 3DS-IVC. (same approach) \cite{main_paper}
    \end{theorem}

\end{frame}



