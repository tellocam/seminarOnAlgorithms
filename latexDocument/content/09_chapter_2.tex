\section{Questions}

\textcolor{red}{
In the end, here will be the answer to the question "Think about a meaningful real-world application for this problem and briefly describe it."\\
1. Think about a meaningful real-world application for this problem and briefly describe it.\\
2. Develop a meaningful deterministic construction heuristic.\\
3. Derive a randomized construction heuristic to be applied iteratively.\\
4. Develop or make use of a framework for basic local search which is able to deal with:\\
• different neighborhood structures\\
• different step functions (first-improvement, best-improvement, random)\\
5. Develop a set of meaningful neighborhood structures to address the different aspects of the problem,
i.e., related to the quality of the s-plexes, or the assignment of nodes to s-plexes.\\
6. Develop or make use of a Variable Neighborhood Descent (VND) framework which uses your
neighborhood structures.\\
7. Implement a Greedy Randomized Adaptive Search Procedure (GRASP) using your randomized
construction heuristic and an effective neighborhood structure with one step function or (a vari-
ant of) your VND. Note that the union of existing neighborhood structures also constitutes a
(composite) neighborhood structure.\\
8. Implement one of the following metaheuristics:\\
• General Variable Neighborhood Search (GVNS) on top of your VND\\
• Simulated Annealing (SA)\\
• Tabu Search (TS)\\
9. Perform some manual tuning of relevant algorithmic parameters to find sensible parameter settings
for the final experiments. Relevant parameters may be related to the degree of randomization,
neighborhood structure sizes, probabilities for the random step function in composite neighborhood
structures, the cooling schedule, the tabu list length and its variation, etc. Report the impact of a
number of different settings on the solution quality of a selected meaningful subset of instances.
10. Use delta-evaluation.\\
Explain which steps in your algorithm use delta-evaluation and describe
why delta-evaluation results in better performance in this step. Are there other elements in your
algorithm that could have also benefitted from delta-evaluation?\\
11. Run experiments and compare all your algorithms on the instances provided in TUWEL:\\
(a) deterministic and randomized construction heuristic and GRASP\\
(b) Use the solution of the deterministic construction heuristic to test the other implementations:\\
i. Local search for at least three selected (possibly composite) neighborhood structures using
each of the three step functions (i.e., at least nine different algorithm variants).\\
ii. VND\\
iii. GVNS, SA, or TS\\
12. Write a report containing the description of your algorithms, the experimental results and what
you conclude from them; see the general information document for more details.}

\pagebreak

\section{Questions to Consider during developtment}

\textcolor{red}{
• How is your solution represented?\\
• Ad 4: How do you generate different solutions? Which parts of your algorithm can be reasonably
randomized and how can you control the degree of randomization?\\
• Ad 3 and 4: Does randomization and iterated application improve the generated solutions?\\
• What parameters do you use and which values do you chose for them?\\
• Can subsequent – possibly non-improving – moves in your neighborhood structures reach every\\
solution in the search space? Or at least one optimal solution?
• Local search: How many iterations does it take on average to reach a local optima? What does
this say about your neighborhood structures?\\
• How does incremental evaluation work for your neighborhood structures?\\
• What is the time complexity to fully search one neighborhood of your neighborhood structures?\\
• VND: Does the order of your neighborhood structures affect the solution quality?}